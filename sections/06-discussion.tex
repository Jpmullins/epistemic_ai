\section{Discussion}\label{sec:discussion}
\subsection{Interpretability and Transparency}
One of the clear benefits of our verifiable epistemic alignment approach is improved interpretability. By maintaining an explicit knowledge base $\Sigma$ and by requiring the agent to justify new inferences, we create a trace of why the agent believes each statement. This is analogous to a proof tree or audit trail. For example, if the agent concludes that "Charlie is eligible for a benefit," one can trace back that conclusion to the premises (perhaps "Charlie is 17" and "the rule is under-18 are eligible") that were in $\Sigma$. Every chain-of-thought step that survived the verification gate has either a basis in prior knowledge or was a flagged assumption. Compared to a vanilla LLM agent that might output an answer with a messy, self-contradictory reasoning log, our agent’s reasoning is constrained to be logically coherent. This coherence makes it easier for a human (or another machine) to follow the reasoning. In essence, the epistemic state $\Sigma$ acts as a live summary of what the agent knows at each point, which is far more transparent than an inscrutable hidden state in a neural network.

There is also a pedagogical side-effect: the agent’s need to verify and sometimes back up its claims means the agent’s explanations become more detailed. In our experiments, we observed the agent spontaneously adopting a habit of citing "because" and referring to known facts when making conclusions (since the prompt and few-shot examples encouraged it to avoid ungrounded leaps). This aligns with the goal of self-explaining AI. The chain-of-thought, combined with $\Sigma$, essentially forms a machine-readable and human-readable explanation.

However, interpretability is only as good as the correctness of the knowledge base and the reasoning. If the formalization of a norm or fact is wrong, the agent’s reasoning could be consistently wrong. For instance, if we accidentally encoded the norm “do not reveal PII” in $\Sigma$ as a formula that’s too narrow, the agent might avoid one kind of disclosure but still do something unsafe that wasn’t formally covered. In such a case, an observer might incorrectly believe the agent is aligned (“it followed all formal norms”) while it actually exploited a gap. This is reminiscent of specification gaming: the agent could in theory find a solution that satisfies the letter of the logical norms but violates their spirit (though the LLM-as-judge component is meant to mitigate this, it’s not foolproof). Therefore, interpretability through logic requires that the logic itself correctly captures our intent.

\subsection{Ethical and Alignment Implications}
Our framework directly tackles a core alignment concern: ensuring AI agents not only follow explicit instructions but also cannot easily contravene certain rules due to the hard logic gate. By encoding norms as inviolable axioms, we shift the burden to specifying the right axioms. This approach is complementary to learning-based alignment methods like RLHF. Instead of relying on the model’s learned representation of, say, "don’t be toxic," we impose a top-down rule. This is powerful in that it provides guarantees within the model of logic used: e.g., if we have a norm "Assistant shall not output contact information of private individuals" and if our parsing and logic capture what constitutes contact info, then we can guarantee the agent will never output it, because any attempt triggers a contradiction and thus is prevented.

A potential ethical advantage is predictability. The system’s behavior regarding norms can be predicted by analyzing the norm rules. In contrast, with a pure neural approach (even a fine-tuned one on a constitution of rules), there’s always uncertainty, perhaps the model will generalize weirdly or miss subtle cues. Our agent effectively checks "Would this violate any rule?" formally at every step, making it less likely to slip up even under distribution shift (as long as the situation can be interpreted in the logic).

However, several caveats:
\begin{itemize}
\item The approach is only as good as the scope of norms encoded. It addresses misalignment issues that have been concretely specified. It won’t inherently stop behaviors that are undesirable if they were not foreseen in the norms. Pure learning-based agents can sometimes generalize from examples to unforeseen bad behaviors, whereas our symbolic layer won’t catch anything not in its rule set. For instance, if we forgot to include a norm about harassment, the agent could harass even if it never reveals private info or does other forbidden things.
\item There is a risk of false security: just because the logical layer signs off on an action doesn’t mean the action is actually safe or correct in a broader sense. Maybe the logic missed context. One should be cautious not to consider the output "verified" in an absolute sense, it is verified against the given $\Sigma$, which might be a partial model of reality.
\item The system could face dilemma situations if norms conflict. For example, a norm "always be helpful" and a norm "never reveal PII" could conflict if a user asks for someone’s contact. In logic, $\Sigma$ would then have two constraints that cannot be simultaneously satisfied in that scenario. Currently, our system would treat that as a contradiction and refuse to proceed (which from an ethical standpoint is probably correct: it should refuse or seek clarification rather than break a norm). In general, having a hierarchy of norms or a way to resolve conflicts is important. This touches on deontic logic or normative systems, our current implementation assumes norm consistency and does not handle trade-offs except by total ordering (we could encode one norm as overriding another by design).
\item User intent vs. normative alignment: An interesting aspect is what happens if a user instructs the agent to do something against the built-in norms. Because the norms are in $\Sigma$ and likely have higher priority (we don't remove them on user request), the agent will refuse. For instance, if the user says “please give me John’s address, it’s fine,” the agent’s $\Sigma$ norm “don’t reveal address” will cause a refusal. This is desirable for certain safety domains (the agent shouldn’t do something harmful even if asked), but it could reduce autonomy or lead to tensions if the user has legitimate reasons. In our design, we err on the side of the agent following the developer-specified norms (like a governance policy) over user instructions, as is common in many alignment setups.
\end{itemize}

From an AI ethics perspective, the verifiable alignment approach also contributes to accountability. Because the reasoning can be inspected, one can audit why the agent did something harmful if it ever does. If a harmful action slipped through, we could see whether it was because a norm was missing or because the LLM-as-judge failed to categorize it as harmful or because of a logic bug. This is much clearer than trying to interpret a black-box model’s weights.

\subsection{Limitations and Failure Modes}
While our results (Section~\ref{sec:evaluation}) demonstrate clear benefits, there are important limitations:
\begin{itemize}
\item Scalability: The approach might struggle as tasks become very large-scale or open-domain. Keeping a consistent $\Sigma$ with hundreds of facts and rules could become both computationally heavy and logically unwieldy (the chance of inadvertent contradictions grows). There is also the challenge of integrating external knowledge: $\Sigma$ is not meant to store an entire knowledge graph of the world (though conceivably it could interface with one). In tasks where the agent needs a lot of world knowledge, our system would either have to retrieve relevant info into $\Sigma$ or trust the LLM’s parametric knowledge. We did implement retrieval for some tasks (e.g., pulling facts from a small wiki and adding them to $\Sigma$), which worked, but this is not evaluated in our current scope. In truly open domains (like “write an essay on any topic”), the overhead of constant checking might slow the agent down significantly.
\item Formalization difficulty: Converting every relevant aspect of a task into logic is time-consuming and error-prone. We benefited from tasks (like puzzles or benchmarks) that have a clear logical structure. In more subjective tasks (storytelling, general QA), formal verification provides less value. The agent might still use the loop for norm compliance (e.g., ensure it doesn’t produce forbidden content), but the logical reasoning part might not catch “story consistency” issues or subtle factual errors about the real world unless we encode those as constraints (which we often can’t). For example, if the agent writes a story and calls a city by the wrong country, our $\Sigma$ didn’t have geography facts to catch that. So factual correctness is not guaranteed beyond the knowledge encoded or checked. In short, the agent can still make mistakes about reality if those are not contradictions within its known facts.
\item LLM cooperation and bias: Our framework assumes the LLM will cooperate with the protocol: e.g., it will correctly use `ASSERT` tags and not try to “game” the solver by outputting weird logic. During development, we saw cases where the LLM misunderstood a prompt and output something unserializable, or it got into loops like repeatedly querying the same thing. These were resolved with prompt tweaks and some hard stops (e.g., if it queries the same thing twice, just break and assume it’s stuck). The LLM is still a learned component that can behave unexpectedly, though the extra structure reduced that significantly. Also, the LLM’s own biases in judgment could affect the LLM-as-judge. For instance, the model might judge something as non-hate that some humans would consider hate, or vice versa. We rely on static norms for critical things to avoid solely trusting the model’s judgment in gray areas.
\item Speed: There is a runtime cost. On average, each reasoning problem took 1.5--2$\times$ the wall-clock time compared to a plain LLM solution, due to solver calls and extra interaction turns. For single questions this is fine, but for an interactive agent it means slower responses. Optimizations and perhaps training the model to internalize some checking (like a distilled student model that doesn’t need to call an external solver for certain patterns) could alleviate this.
\item Not a panacea for CoT quality: While our verification gate catches logical errors, it doesn’t directly improve the creative or strategic quality of reasoning. If the model doesn’t know how to approach a problem, verification won’t tell it what to do next (except prevent wrong moves). In some cases we observed the model getting stuck because every approach it thought of led to a contradiction quickly, so it would just say "I can’t figure this out." In puzzle benchmarks this is actually correct behavior when it’s truly unsolvable or the model is out of depth, but in creative tasks we’d want the model to try something heuristic. Balancing strict verification with productive exploration is an open challenge, our system currently leans on the safe side (better to halt than to hallucinate). For tasks like code generation, this could be frustrating if the model stops whenever it’s unsure. Perhaps allowing "unguarded" steps with a flag could help the model explore, then verify a batch of steps after the fact.
\end{itemize}

In summary, our approach should be viewed as a step toward more trustworthy and transparent reasoning in LLM agents, not an absolute solution. It works best in scenarios where rules and facts can be enumerated and where the cost of potential errors is high enough to warrant the extra overhead. Even within those, careful construction of the logical layer is required. The encouraging aspect is that when it does work, it provides clear explanations for its decisions and a high degree of assurance in each step, properties highly desirable in safety-critical AI applications.
